\documentclass{article}


\begin{document}
\newcommand{\PLtitle}{\setlength{\parindent}{0pt}
\begin{center}

  \huge{\textbf{\Ltitle\\}}
  \normalsize 

  CSE 3053 Introduction to Computer Networks \\
  Date: \date\\
  Name: Kassidy Maberry\\
\end{center}
}

\def\Ltitle{Homework 2}
\def\author{Kassidy Maberry}
\def\date{2024/09/06}
\def\subP{\hspace{20mm}}

\PLtitle

\section*{Problem 1} % Done
Given a frequency of 450 KHz, calculate the corresponding time period.\\\\

To determine the period we will use the following formula:\\
$T = \frac{1}{f}$.\\
Plugging in $T = \frac{1}{450 * 10^{3}}$.\\
Giving us the result of $2.22 *10^{-6}$s or a final result of\\
$2.22\mu$s

\section*{Problem 2} % Done
Given a time period of 420 ns, calculate the corresponding frequency.\\\\
To determine the period we will use the following formula:\\
$f = \frac{1}{T}$.\\
$f = \frac{1}{420 * 10^{-9}}$
$f = 2.4 * 10^{6}$.\\
Giving us a final result of $2.4$MHz.\\

\section*{Problem 3} % Done
What is the bit rate for each of the following signals?\\\\
\hspace{20mm}a. A signal in which 1 bit lasts 0.00003s.\\
To determine the bit rate we will need to first determine the baud rate which is the number of distinct changes over the medium 
per sec. We can derive the following formula:\\
$b = \frac{1}{0.00003}$.\\
$b = 33,333$ bauds.\\
Now to get the bit rate we need to multiply by the number of bits per change.\\
$33,333 * 1 = 33,333$.\\
Giving us a final bitrate of 33.333 kb/s.\\
\hspace{20mm}b. A signal in which 25 bits last 350 $\mu$s.\\
To determine the bit rate we will need to first determine the baud rate which is the number of distinct changes over the medium 
per sec. We can derive the following formula:\\
$b = \frac{1}{350 * 10^{-6}}$.\\
$b = 2,860$ bauds.\\
Finally we need to multiply by the bits per change.\\
$b_{r} = 2,860 * 25$.\\
$b_{r} = 71.5$ kb/s.\\

\section*{Problem 4} % Done
A device is transmitting data at a rate of 2500 bps.\\\\
\subP How long does it take to send out a single character (8 bits).\\
To determine this we will use the following formula.\\
$t = \frac{s*c}{r}$.\\
Where t is the time, s is the total amount of characters, c is the size of a character,
and r is the rate of data transmission.\\
Plugging in we get\\
$t = \frac{8}{2500}$\\
$t = 0.0032$s
or a final result of $3.2$ms.\\\\


\subP How long does it take to send a file of 250,000 characters (each
character is 8 bits)?\\
To determine this we will use the following formula.\\
$t = \frac{s*c}{r}$.\\
Where t is the time, s is the total amount of characters, c is the size of a character,
and r is the rate of data transmission.\\
Plugging in we get\\
$t = \frac{250,000 * 8}{2500}$.\\
$t = 800$s.\\

\section*{Problem 5} % Done
What is the theoretical capacity of a channel in each of the following cases:\\\\
\subP a. Bandwidth: 350 KHz, SNRdB = 50.\\
To determine this we will use the following equation:\\
$c = BW * log_{2}(1 + 10^{(SNRdb)/10})$.\\
$c = 350 * 10^{3} * log_{2}(1 + 10^{50/10})$.\\
$c = 350 * 10^{3} * log_{2}(1 + 10^{5})$.\\
$c = 350 * 10^{3} * 13.288$.\\
$c = 4650 * 10^{3}$.\\
Giving us a final capacity of 4.65 Mb/s

\subP b. Bandwidth: 17 MHz, SNRdB = 25.\\
$c = BW * log_{2}(1 + 10^{(SNRdb)/10})$.\\
$c = 17 * 10^{6} * log_{2}(1 + 10^{25/10})$.\\
$c = 17 * 10^{6} * log_{2}(1 + 10^{2.5})$.\\
$c = 17 * 10^{6} * 8.308$.\\
$c = 141.2 * 10^{6}$.\\
Giving us a final capacity of 141.2 Mb/s


\section*{Problem 6}
What is the approximate optimal SNR of a 8-QAM, 240 channel in order to achieve its
maximum capacity C bit/sec?\\\\


\section*{Problem 7} % Done
What is the max bit-rate capacity of a 4500 Hz, 4096-QAM encoded channel with a SNR of
45 dB?\\
$c_{ny} = 2B log_{2}(L)$.\\
$c_{shan} = Blog_{2}(1 * 10^{SNR_{db}/10})$.\\
Plugging in\\
$c_{ny} = 2 * 4500 * log_{2}(4096)$.\\
$c_{ny} = 9000 * 12$\\
$c_{ny} = 108,000$b/s\\
$c_{shan} = 4500 * log_{2}(1 + 10^{45/10})$.\\
$c_{shan} = 4500 * 14.95$.\\
$c_{shan} = 67,275$b/s.\\
Since $c_{shan} < c_{ny}$ our max-bit rate capacity is $67,275$b/s.\\

\section*{Problem 8}
Compute the latency of a 126-bit frame that is being sent over a 6000 Km link with 30
routers, given the following:\\
\subP Router Information:\\
\subP a. Queuing time = 160 $\mu$s.\\
\subP b. Processing time = 10 $\mu$s.\\
Link Characteristics:\\
\subP a. Bandwidth = 45 Mb/sec.\\
\subP b. Propagation speed of the link = $4.5x10^8$ m/s.\\

\section*{Problem 9} % Done
What is the Nyquist sampling rate for each of the following signals?\\
\subP a. A low-pass signal with bandwidth of 35 Khz.\\

For a low pass filter we can use the following equation:\\
$N = 2*f_{m}$.\\
Where $f_{m} = BW$.\\
Plugging in we get $N = 2*35 * 10^{3}$.\\
$N = 70,000$ sample/s\\

\subP b. A band-pass signal with bandwidth of 350 KHz if the lowest
frequency is 250 KHz.\\
For a band pass filter we can use the following equation:\\
$N = 2*f_{m}$.\\
Where $f_{m}$ is the highest frequency of the signal.\\
Plugging in we get $N = 2*(250 + 350) * 10^{3}$.\\
$N = 1,200,000$ sample/s\\


\section*{Problem 10}
Given three approaches for digital encoding: 8B6T, 8B6Q, and 8B/10B, answer the
following questions:\\
\subP a. Which one will have the best utilization of the channel bandwidth? Justify your answer.\\
\subP b. Which one requires an additional encoding stage before transmitting codes? Justify your
answer.\\

\section*{Problem 11} % Done
Assume that a voice channel occupies a bandwidth of 4 kHz. We need to multiplex 30 voice
channels with guard bands of 500 Hz separating the channels using FDM. Calculate the total
required bandwidth.\\
To determine the required bandwidth we will use the following equation:\\
$BW = 30(BW_{c} BW_{g})$.\\
$BW = 30(4000 + 500)$.\\
$BW = 30 * 4500$.\\
$BW = 135,000$.\\
Giving our final result of $135$kHz.\\

\section*{Problem 12} % NOT STARTED; Remember BW = 1.544 Mb/s
Answer the following questions about a T-1 line:\\\\
\subP a. What is the duration of a frame?\\
\subP b. What is the overhead (number of extra bits per second)?\\

\section*{Problem 13}
We have a digital medium with a data rate of 15 Mbps. How many 40-kbps voice channels can be
carried by this medium if we use DSSS with the Barker sequence?\\\\

\section*{Problem 14}
A path in a digital circuit-switched network has a data rate of 2.5 Mbps. The exchange of 2000 bits
is required for the setup and tear down phases. The distance between the two parties is 6500 km.
Answer the following questions if the propagation speed is $2.5x10^8$ m/s:\\

\subP a. What is the total delay if 1500 bits of data are exchanged during the data transfer phase?\\
To determine the total delay we will use the following formula:\\
$d_{cs} = s + m_{message} + t$.\\
Where s is the circuit set up time, m is the message transmission time, and t is the propagation delay.\\
$s = 2(a + t)$.\\
where a is the message transmission of the bits required to set up the network.\\
$m = \frac{b}{n}$.\\
Where b is the size of our message and n is the data rate. Plugging in we get.\\
$m = \frac{2000}{2.5*10^{6}} = 0.8 * 10^{-3}$s.\\
$t = \frac{x}{c}$.\\
Where x is the distance and c is the propagation speed.\\
$t = \frac{6,500 * 10^{3}}{2.5*10^{8}}$.\\
$t = 2.6 * 10^{-2}$s.\\
Finally we can plug into s and obtain.\\
$s = 2 * 2.68 * 10^{-2} = 5.36*10^{-2}$s.\\
Lastly we just have to determine $m_{message}$.\\
$m_{message} = \frac{b}{n}$.\\
Where b is the size of our message and n is the data rate. Plugging in we get.\\
$m_{message} = \frac{1500}{2.5 * 10^{-6}} = 0.06 * 10^{-2}$s.\\
$s = (5.36 + 2.6 + 0.06)*10^{-2} = 8.02 * 10^{-2}$s.\\
Giving a total delay of 80.2ms.\\

\subP b. What is the total delay if 120,000 bits of data are exchanged during the data transfer
phase?\\
$d_{cs} = s + m_{message} + t$.\\
Where s is the circuit set up time, m is the message transmission time, and t is the propagation delay.\\
$s = 2(a + t)$.\\
where a is the message transmission of the bits required to set up the network.\\
$m = \frac{b}{n}$.\\
Where b is the size of our message and n is the data rate. Plugging in we get.\\
$m = \frac{2000}{2.5*10^{6}} = 0.8 * 10^{-3}$s.\\
$t = \frac{x}{c}$.\\
Where x is the distance and c is the propagation speed.\\
$t = \frac{6,500 * 10^{3}}{2.5*10^{8}}$.\\
$t = 2.6 * 10^{-2}$s.\\
Finally we can plug into s and obtain.\\
$s = 2 * 2.68 * 10^{-2} = 5.36*10^{-2}$s.\\
Lastly we just have to determine $m_{message}$.\\
$m_{message} = \frac{b}{n}$.\\
Where b is the size of our message and n is the data rate. Plugging in we get.\\
$m_{message} = \frac{120000}{2.5 * 10^{-6}} = 4.8 * 10^{-2}$s.\\
$s = (5.36 + 2.6 + 4.8)*10^{-2} = 12.76 * 10^{-2}$s.\\
Giving a total delay of 128ms.\\

\section*{Problem 15}
Answer the following questions:\\
\subP a. Can a routing table in a datagram network have two entries with the same
destination address? Explain.\\
\subP b. Can a switching table in a virtual-circuit (VC) network have two entries with
the same:\\
\subP\subP i) input port number?\\
\subP\subP ii) output port number?\\
\subP\subP iii) incoming VCIs?\\
\subP\subP iv) outgoing VCIs?\\
\subP\subP v) incoming values (port, VCI)?\\
\subP\subP vi) outgoing values (port, VCIs)?\\

\section*{Problem 16}
Is circuit-switching (CS) always faster than datagram (DG) in packet switching over the
subnet? Justify your answer. (Hint: compare their delay equations).\\

\section*{Problem 17}
What is the difference between random access protocols, channelizing protocols, and
controlled access protocols? Briefly compare and contrast the three protocol types.\\
\end{document}